%%%%%%%%%%%%%%%%%%%%%%%%%%%%%%%%%%%
%This is the LaTeX ARTICLE template for RSC journals
%Copyright The Royal Society of Chemistry 2016
%%%%%%%%%%%%%%%%%%%%%%%%%%%%%%%%%%%

\documentclass[twoside,twocolumn,9pt]{article}
\usepackage{extsizes}
\usepackage[super,sort&compress,comma]{natbib} 
\usepackage[version=3]{mhchem}
\usepackage[left=1.5cm, right=1.5cm, top=1.785cm, bottom=2.0cm]{geometry}
\usepackage{balance}
\usepackage{times,mathptmx}
\usepackage{sectsty}
\usepackage{graphicx} 
\usepackage{lastpage}
\usepackage[format=plain,justification=justified,singlelinecheck=false,font={stretch=1.125,small,sf},labelfont=bf,labelsep=space]{caption}
\usepackage{float}
\usepackage{fancyhdr}
\usepackage{fnpos}
\usepackage[english]{babel}
\usepackage{amsmath}
\addto{\captionsenglish}{%
  \renewcommand{\refname}{Notes and references}
}
\usepackage{array}
\usepackage{droidsans}
\usepackage{charter}
\usepackage[T1]{fontenc}
\usepackage[usenames,dvipsnames]{xcolor}
\usepackage{setspace}
\usepackage[compact]{titlesec}
\usepackage{hyperref}
%%%Please don't disable any packages in the preamble, as this may cause the template to display incorrectly.%%%

\newcommand*{\citen}[1]{%
	\begingroup
	\romannumeral-`\x % remove space at the beginning of \setcitestyle
	\setcitestyle{numbers}%
	\cite{#1}%
	\endgroup   
}

\usepackage{epstopdf}%This line makes .eps figures into .pdf - please comment out if not required.

\definecolor{cream}{RGB}{222,217,201}

\begin{document}

\pagestyle{fancy}
\thispagestyle{plain}
\fancypagestyle{plain}{

%%%HEADER%%%
\fancyhead[C]{\includegraphics[width=18.5cm]{head_foot/header_bar}}
\fancyhead[L]{\hspace{0cm}\vspace{1.5cm}\includegraphics[height=30pt]{head_foot/journal_name}}
\fancyhead[R]{\hspace{0cm}\vspace{1.7cm}\includegraphics[height=55pt]{head_foot/RSC_LOGO_CMYK}}
\renewcommand{\headrulewidth}{0pt}
}
%%%END OF HEADER%%%

%%%PAGE SETUP - Please do not change any commands within this section%%%
\makeFNbottom
\makeatletter
\renewcommand\LARGE{\@setfontsize\LARGE{15pt}{17}}
\renewcommand\Large{\@setfontsize\Large{12pt}{14}}
\renewcommand\large{\@setfontsize\large{10pt}{12}}
\renewcommand\footnotesize{\@setfontsize\footnotesize{7pt}{10}}
\makeatother

\renewcommand{\thefootnote}{\fnsymbol{footnote}}
\renewcommand\footnoterule{\vspace*{1pt}% 
\color{cream}\hrule width 3.5in height 0.4pt \color{black}\vspace*{5pt}} 
\setcounter{secnumdepth}{5}

\makeatletter 
\renewcommand\@biblabel[1]{#1}            
\renewcommand\@makefntext[1]% 
{\noindent\makebox[0pt][r]{\@thefnmark\,}#1}
\makeatother 
\renewcommand{\figurename}{\small{Fig.}~}
\sectionfont{\sffamily\Large}
\subsectionfont{\normalsize}
\subsubsectionfont{\bf}
\setstretch{1.125} %In particular, please do not alter this line.
\setlength{\skip\footins}{0.8cm}
\setlength{\footnotesep}{0.25cm}
\setlength{\jot}{10pt}
\titlespacing*{\section}{0pt}{4pt}{4pt}
\titlespacing*{\subsection}{0pt}{15pt}{1pt}
%%%END OF PAGE SETUP%%%

%%%FOOTER%%%
\fancyfoot{}
\fancyfoot[LO,RE]{\vspace{-7.1pt}\includegraphics[height=9pt]{head_foot/LF}}
\fancyfoot[CO]{\vspace{-7.1pt}\hspace{13.2cm}\includegraphics{head_foot/RF}}
\fancyfoot[CE]{\vspace{-7.2pt}\hspace{-14.2cm}\includegraphics{head_foot/RF}}
\fancyfoot[RO]{\footnotesize{\sffamily{1--\pageref{LastPage} ~\textbar  \hspace{2pt}\thepage}}}
\fancyfoot[LE]{\footnotesize{\sffamily{\thepage~\textbar\hspace{3.45cm} 1--\pageref{LastPage}}}}
\fancyhead{}
\renewcommand{\headrulewidth}{0pt} 
\renewcommand{\footrulewidth}{0pt}
\setlength{\arrayrulewidth}{1pt}
\setlength{\columnsep}{6.5mm}
\setlength\bibsep{1pt}
%%%END OF FOOTER%%%

%%%FIGURE SETUP - please do not change any commands within this section%%%
\makeatletter 
\newlength{\figrulesep} 
\setlength{\figrulesep}{0.5\textfloatsep} 

\newcommand{\topfigrule}{\vspace*{-1pt}% 
\noindent{\color{cream}\rule[-\figrulesep]{\columnwidth}{1.5pt}} }

\newcommand{\botfigrule}{\vspace*{-2pt}% 
\noindent{\color{cream}\rule[\figrulesep]{\columnwidth}{1.5pt}} }

\newcommand{\dblfigrule}{\vspace*{-1pt}% 
\noindent{\color{cream}\rule[-\figrulesep]{\textwidth}{1.5pt}} }

\makeatother
%%%END OF FIGURE SETUP%%%

%%%TITLE, AUTHORS AND ABSTRACT%%%
\twocolumn[
  \begin{@twocolumnfalse}
\vspace{3cm}
\sffamily
\begin{tabular}{m{4.5cm} p{13.5cm} }

\includegraphics{head_foot/DOI} & \noindent\LARGE{\textbf{Bayesian determination of the effect of a deep eutectic solvent on the structure of lipid monolayers}} \\%Article title goes here instead of the text "This is the title"
\vspace{0.3cm} & \vspace{0.3cm} \\

 & \noindent\large{Andrew R. McCluskey,\textit{$^{ac}$}$^{\dag}$ Adrian Sanchez-Fernandez,\textit{$^{ab}$}$^{\dag\P}$ Karen J. Edler,\textit{$^{a}$}$^{\ast}$ Andrew J. Jackson,\textit{$^{bd}$} Richard A. Campbell,\textit{$^{ef}$} and Thomas Arnold\textit{$^{abcg}$}$^{\ast}$} \\%Author names go here instead of "Full name", etc.

\includegraphics{head_foot/dates} & \noindent\normalsize{Deep eutectic solvents present a novel class of non-aqueous room temperature solvent with tunable properties, that are capable of promoting self-assembly of surfactant molecules. However, the solvation model in these systems still challenges the classic understanding of amphiphilicity. In this work, we present the first example of the self-assembly of phospholipid monolayers at the interface between air and a non-aqueous solvent. Furthermore, we use novel, chemically-consistant Bayesian modelling of reflectometry measurements to show the ability of the deep eutectic solvent to interact with the phosphatidylglycerol lipid head group, leading to an apparent increase in the component volume compared to that observed in water. No such change was observed for the phosphocholine head group, indicating that the interaction is head group specific.} \\

\end{tabular}

	\end{@twocolumnfalse} \vspace{0.6cm}

  ]
%%%END OF TITLE, AUTHORS AND ABSTRACT%%%

%%%FONT SETUP - please do not change any commands within this section
\renewcommand*\rmdefault{bch}\normalfont\upshape
\rmfamily
\section*{}
\vspace{-1cm}


%%%FOOTNOTES%%%

\footnotetext{\textit{$^{a}$~Department of Chemistry, University of Bath, Claverton Down, Bath, BA2 7AY, UK.}}
\footnotetext{\textit{$^{b}$~European Spallation Source, SE-211 00 Lund, Sweden.}}
\footnotetext{\textit{$^{c}$~Diamond Light Source, Harwell Campus, Didcot, OX11 0DE, UK.}}
\footnotetext{\textit{$^{d}$~Department of Physical Chemistry, Lund University, SE-211 00 Lund, Sweden.}}
\footnotetext{\textit{$^{e}$~Institut Laue-Langevin, 71 avenue des Martyrs, 38000, Grenoble, France.}}
\footnotetext{\textit{$^{f}$~Division of Pharmacy and Optometry, University of Manchester, Manchester, M13 9PT, UK.}}
\footnotetext{\textit{$^{g}$~ISIS Neutron and Muon Source, Science and Technology Facilities Council, Rutherford Appleton Laboratory, Harwell Oxford, Didcot OX11 0QX, UK.}}
\footnotetext{\textit{$^{\P}$~Present address: Department of Food Technology, Lund University, SE-211 00 Lund, Sweden}}
\footnotetext{$^{\ast}$~Corresponding author: k.edler@bath.ac.uk, tom.arnold@esss.se}


%Please use \dag to cite the ESI in the main text of the article.
%If you article does not have ESI please remove the the \dag symbol from the title and the footnotetext below.
\footnotetext{\dag~Electronic Supplementary Information (ESI) available: A series Jupyter notebooks and Python plotting scripts, allowing for a fully reproducible analysis of all data presented herein as well as figures showing the probability distribution functions for each of the lipid models. See DOI: 10.1039/b000000x/}
%additional addresses can be cited as above using the lower-case letters, c, d, e... If all authors are from the same address, no letter is required

\footnotetext{$^{\dag}$~These authors have contributed equally to the work presented within.}
%\footnotetext{\ddag~Additional footnotes to the title and authors can be included \textit{e.g.}\ `Present address:' or `These authors contributed equally to this work' as above using the symbols: \ddag, \textsection, and \P. Please place the appropriate symbol next to the author's name and include a \texttt{\textbackslash footnotetext} entry in the the correct place in the list.}


%%%END OF FOOTNOTES%%%

%%%MAIN TEXT%%%%
\section{Introduction}
Deep eutectic solvents (DES) are green, sustainable solvents obtained through the complexation of naturally occurring compounds, such as sugars, alcohols, amines and carboxylic acids, among others.\cite{Smith2014, Dai2013} An extensive hydrogen-bonding network is present between these precursors, allowing the mixture to remain liquid at room temperature due to its high-entropic state.\cite{Hammond2016, Hammond2017, Araujo2017} Additionally, through different combinations of the precursors materials, it is possible to tune the physicochemical properties of the solvent, such as polarity,\cite{Pandey2014} viscosity and surface tension,\cite{Smith2014} network charge,\cite{Zahn2016} and hydrophobicity.\cite{Ribeiro2015,vanOsch2015} 

These solvents have recently shown the ability to promote the self-assembly of surfactants into micellar structures\cite{Sanchez-Fernandez2016,Arnold2015} and to stabilise the conformation of non-ionic polymer species,\cite{Sapir2016} indicating the presense of a solvophobic effect. The behaviour and conformation of biomolecules in DES have seen an increase in interest,\cite{Esquembre2013,Gorke2010,Gorke2008,Monhami2014,Wu2014,Harifi-Mood2017,Milano2017,Sanchez-Fernandez2017} due to potential applications in the preservation of biomolecules, as environments for enzymatic reactions.\cite{Merza2018} Futhermore, recent investigations have also shown that DES have been able to support the formation of phospholipid bilayers.\cite{Bryant2017,Bryant2016,Gutierrez2009} 

The formation of phospholipid monolayers plays a key role in many biological and technological processes. Amphiphilic lipids commonly show a low soluibilty in the one of the two phases, leading to the formation of a stable monolayer at the interface.\cite{Mohwald1990} Phospholipids consist of a charged headgroup, either anionic or zwitterionic, and investigations at the air-salt water interface have revealed the importance of the lipid-ion interactions on structure, monomer packing, and stability of the monolayer.\cite{Mohwald1990,Kewalramani2010} Despite the broad interest in these systems, the presence of stable phospholipid monolayers in non-aqueous media has not been previously reported, to the best of the authors' knowledge. 

Recent developments in computational resource and software have enabled powerful methodologies and algorithms to be harnessed by those from non-expert backgrounds. This has mostly occurred through open-source software projects such as the Python language and the Jupyter notebooks framework.\cite{vanRossum1995,Kluyver2016} In the area of neutron and X-ray reflectometry data-analysis, this has led to the development of refnx,\cite{Nelson2018} a Python library for the fitting of layer-based models to reflectometry data. The refnx library enables the use of custom models that contain chemically-relevant information. This information includes constaints such as that the number density of phospholipid headgroups must be the same as the number density of pairs of phospholipid tailgroups and that the length of the tail chains should not surpass a maximum of the Tanford length for a given chain.\cite{Tanford1980} This length is an emperically calculated value for the maximum expansion of a series of carbon atoms. 

The use of a Python library for fitting enables powerful probability distribution function (PDF) sampling methods to be used such as the Goodman \& Weare Affine Invariant Markov chain Monte Carlo (MCMC) Ensemble,\cite{Goodman2010} as implemented in the Python library emcee.\cite{Foreman-Mackey2013} This method allows for the Bayesian sampling of a high-dimensionality parameter space, such as that which is relevant to reflectometry fitting, to be performed with relative ease. This results in estimations of the inverse uncertainties associated with each parameter as well as information about the correlations between different parameters within the model. 

In this work, we present the first investigation of the structure of phospholipid monolayers at the air-DES interface, as determined by chemically-consistant modelling of X-ray reflectometry (XRR) measurements. Four different phospholipids; 1,2-dipalmitoyl-sn-glycero-3-phosphocholine (DPPC), 1,2-dimyristoyl-sn-glycero-3-phosphocholine (DMPC),  1,2-dilauroyl-sn-glycero-3-phosphocholine (DLPC) and 1,2-dimyristoyl-sn-glycero-3-phospho-(1'-rac-glycerol) (DMPG), were studied at the interface between a 1:2 mixture of choline chloride:glycerol and air. This allowed the nature of two, chemically distinct, phospholipid headgroups to be understood in this non-aqueous solvent, in addition to the effect of the tail chain length. The analysis was then extended to model complementary neutron reflectometry measurements for two contrasts of DMPC and DPPC at a single surface pressure. 

\section{Experimental}

\subsection{Materials}
Choline chloride (99 \%, Sigma-Aldrich) and glycerol (99 \%, Sigma-Aldrich) d$_9$-choline chloride (99 \%, 98 \% D, CK Isotopes) and d$_8$-glycerol (99 \%, 98 \% D, CK Isotopes)  were purchased and used without further purification. The DES was prepared by mixing the precursors at the appropriate mole ratio, and heating at 80 $^\circ$C until a homogeneous, transparent liquid formed.\cite{Smith2014} The solvent was equilibrated overnight at 40 $^\circ$C and subsequently stored under a dry atmosphere. Due to the limited availability of the deuterated precursors, a fully protonated subphase (hDES) and a partially deuterated subphase (hdDES) were prepared and used during the neutron reflectometry (NR) experiment. The partially deuterated subphase was prepared using the following mixtures of precursors: 1 mole of 0.38 fraction of h-choline chloride/0.62 mole fraction of d-choline chloride; and 2 moles of 0.56 mole fraction of h-glycerol/0.44 mole fraction of d-glycerol. The solvent was subsequently prepared following the procedure discussed above. 

The water content of the DES was determined before and after each experiment by Karl-Fischer titration (Mettler Toledo DL32 Karl-Fischer Coulometer, Aqualine Electrolyte A, Aqualine Catholyte CG A) in order to ensure water presence was kept to a minimum. Those measurements showed that the water content of the solvent was kept below 0.3 wt\% during all the experimental procedures presented here, which we assume to be negligible and have little impact on the characteristics of the DES.\cite{Hammond2016,Hammond2017}

DPPC (> 99 \%, C$_{16}$ tails), DMPC (> 99 \%, C$_{14}$ tails), and DMPG (> 99 \%, C$_{14}$ tails) were supplied by Avanti Polar Lipids and DLPC (> 99 \%, C$_{12}$ tails) was supplied by Sigma-Aldrich and all were used as received. Deuterated versions of DPPC (d$_{62}$-DPPC, > 99 \%, deuterated tails-only) and DMPC (d$_{54}$-DPPC, > 99 \%, deuterated tails-only) were supplied by Avanti Polar Lipids and used without further purification. These phospholipids were dissolved in chloroform (0.5 mg/mL) at room temperature. 

In the XRR experiment, sample preparation was performed in situ using the standard method for the spreading of insoluble monolayers on water: a certain amount of the phospholipid solution was spread onto the liquid surface in order to provide a given surface concentration. After the evaporation of the chloroform, it is assumed that the resulting system is a solvent subphase with a monolayer of phospholipid at the interface. Surface concentration was modified by closing and opening the PTFE barriers of a Langmuir trough. In order to minimise the volumes used in the NR experiment (to keep the cost of deuterated compounds to a managageable level) it was not possible to use a Langmuir trough. Instead, small Delrin adsorption troughs were used that did not have controlable barriers. This resulted in no control over the surface pressure of the measurement and therefore it was not possible to co-refine different NR contrasts. 

\subsection{Methods}
XRR measurements were taken on I07 at Diamond Light Source, at 12.5 keV photon energy using the double-crystal-deflector.\cite{Arnold2012} The reflected intensity was measured in a momentum transfer range from 0.018 to 0.7 \AA$^{-1}$. The data were normalised with respect to the incident beam and the background was measured from off-specular reflection and subsequently subtracted. Samples were equilibrated for at least one hour and preserved under an argon atmosphere to minimise the adsorption of water by the subphase. XRR data were collected for each of the lipids, DMPC, DPPC, DLPC and DMPG at two surface pressures, 20 mNm$^{-1}$ and 30 mNm$^{-1}$, as measured with an aluminium Wilhelmy plate; all measurements were made at 22 $^\circ$C. 

The NR experiments were performed on FIGARO at the Institut Laue-Langevin using the time-of-flight method.\cite{Campbell2011} Data at two incident angles of 0.62$^\circ$ and 3.8$^\circ$ were measured to provide a momentum transfer range from 0.005 to 0.18 \AA$^{-1}$. A single surface concentration for each system and contrast was measured. Similar to the X-ray procedure, samples were given enough time to equilibrate (at least two hours) and kept under an inert atmosphere.

\subsection{Data analysis}
The use of reflectometry to analyse the structure of phospholipids on the surface of water has a history extending over many years.\cite{Mohwald1990,Kewalramani2010,Bayerl1990,Johnson1991,Clifton2012,Helm1987,Daillant1990,Campbell2018} This has led to some variation in the models used to fit experimental data, shown in Table \ref{tab:water}. There appears to be a general consensus that the component volume of the phosphocholine (PC) headgroup is in the range from 320 \AA$^3$ to 360 \AA$^3$ while data from a single source gives the phosphatidylglycerol (PG) headgroup to be 291 \AA$^3$, and as a result these volumes is often used as physical constraints on the layer model when fitting reflectometry data. However, since the current work involves a non-aqueous solvent we do not know whether the head group component volumes used in the literature that are derived from water-based measurements will be appropriate for this work. The charged nature of the zwitterionic and anionic lipid heads means that they are likely to have different interactions with neutral water as compared to the charged DES.\cite{Sanchez-Fernandez2018} Further it has been shown that in the liquid condensed phase, which is present at high surface pressure, the lipid tails will become compressed resulting in a component volume less than measured in the liquid expanded phase.\cite{Campbell2018}

\begin{table*}
	\small
	\caption{\ Lipid component volumes extracted from different literature sources. $V_l$ corresponds to the total lipid volume, MD to molecular dynamics simulation, WAXS to wide-angle X-ray scattering, NB to neutral buoyancy and DVTS to differential vibrating tube densimetry}
	\label{tab:water}
	\begin{tabular*}{\textwidth}{@{\extracolsep{\fill}}lllllllll}
		\hline 
		Lipid & DPPC & & & & DMPC & & & DMPG \\
		\hline
		Reference & [\citen{Armen1998}] & [\citen{Petrache1997}] & [\citen{Sun1994}] & [\citen{Tardieu1973}] & [\citen{Kucerka2004}] & [\citen{Nagle1978}] & [\citen{Schmidt1985}] & [\citen{Pan2012}] \\
		$V_l$/\AA$^3$ &1216.96 & 1219 & 1148 & 1224 & 1101 & 1061 & 1094 & 1058 \\
		$V_h$/\AA$^3$ & 326.00 & 324 & 319 & 360 & 319 & 344 & & 291 \\
		$V_t$/\AA$^3$ & 890.96 & 895 & 829 & 864 & 782 & 717 & & 767 \\
		Method & MD & MD & WAXS & NB & NB & NB & DVTS & DVTS \\
		T/$^\circ$C & 25 & 50 & 24 & 25 & 30 & 30 & & 30 \\
		\hline
	\end{tabular*}
\end{table*}

To allow for the use of a chemically-consistant model, where the lipid component volumes were allowed to vary, the Python library refnx\cite{Nelson2018} as used. This software allows the inclusion of a custom model from which the parameters to be fed into the Abel\`{e}s model for the reflection of light at a given number of stratified interfaces,\cite{Abeles1950,Parratt1954} that is typical for reflectometry fitting, are obtained. This custom model, along with a series of Jupyter notebooks showing, in full, the analysis performed, can be found in the ESI and is available under an MIT license.\cite{mccluskey_2018} 

This chemically-consistent model involves two layers consisting of head groups at the interface with the solvent and tail groups at the interface with the air. The head groups have a calculated scattering length, $b_h$, (found as a summation of the number of electrons in the head group multiplied by the classical radius of the electron), and a component volume, $V_h$. These head groups make up a layer with a given thickness, $d_h$, and roughness, $\sigma_h$, within which some volume fraction of solvent can intercalate, $\phi_h$. The tail groups also have a calculated scattering length, $b_t$, and a component volume, $V_t$, however the thickness of the tail group layer, $d_t$, is found from the length of the carbon tail, $t_t$, and angle that the chain is tilted by with respect to the interface normal, $\theta_t$, 
\begin{equation}
\label{equ:tl}
d_t = t_t \cos{\theta_t}.
\end{equation}
The scattering length density (SLD) of the tail and head layers used in the Abel\`{e}s model can therefore be found as follows, 
\begin{equation}
\text{SLD}_i = \frac{b_i}{V_i}(1 - \phi_i) + \text{SLD}_{s}\phi_i,
\end{equation}
where $\text{SLD}_{s}$ is the scattering length density of the subphase (DES) for the tail and head groups respectively, and $i$ indicates either the tail or head layer, not that it is assumed that no solvent penetrates into the tail layer, e.g. $\phi_t = 0$. To ensure that the number density of head groups and pairs of tail groups is the same, the following constraint was added to the model,\cite{Braun2017}
\begin{equation}
\label{equ:ht}
d_h = \frac{V_hd_t(1-\phi_t)}{V_t(1-\phi_h)}. 
\end{equation}
A single value for the interfacial roughness was fitted for all interfaces, as there is only a single lipid molecule type in each monolayer.

In the first of two steps, this custom model was used to co-refine the component volume of the lipid head group, $V_h$, and the volume of the tail group, $V_t$, across the two surface concentration XRR measurements. The following parameters were allowed to vary; $\theta_t$, $d_h$, $\phi_h$, and $\sigma_{t,h,s}$, independently across the two surface pressurse. While others, shown in Table \ref{tab:invariant}, were held constant at the values given. The length of the carbon chain was kept constant to the value determined by the Tanford equation,\cite{Tanford1980} this is valid due to the condensed nature of the monolayer at this surface concentration resulting in the extended, staggered conformation of the chain being likely. In total for each co-refinement of two XRR measurements, there were ten degrees of freedom in the fitting process. 

\begin{table}[h]
	\small
	\caption{\ The invariant parameters within the chemically-sensible model. 
	$^a$Values obtained from the Tanford formula where the carbon chains are assumed to be fully extended.\cite{Tanford1980} $^b$Values extracted from Sanchez-Fernandez \emph{et al.}\cite{Sanchez-Fernandez2016}}
	\label{tab:invariant}
	\begin{tabular*}{0.48\textwidth}{@{\extracolsep{\fill}}lllll}
		\hline
		Component & $b_t$/fm & $b_h$/fm & $t_t$/\AA & $\text{SLD}$/$\times10^{-6}$\AA$^{-2}$ \\
		\hline
		X-ray & & & & \\
		DLPC & 5073 & 4674 & 15.5$^a$ & -- \\
		DMPC & 5985 & 4674 & 18.0$^a$ & -- \\
		DPPC & 6897 & 4674 & 20.5$^a$ & -- \\
		DMPG & 5985 & 4731 & 18.0$^a$ & --\\
		Air & -- & -- & -- & 0\\
		DES & -- & -- & -- & 10.8$^b$ \\
		\hline
		Neutron & & & & \\
		d$_{54}$-DMPC & 5329.8 & 602.7 & 18.0$^a$ & -- \\
		d$_{62}$-DPPC & 6129.2 & 602.7 & 20.5$^a$ & -- \\
		h-DES & -- & -- & -- & 0.43$^b$  \\
		hd-DES & -- & -- & -- & 3.15$^b$ \\
		\hline
	\end{tabular*}
\end{table}

In the second step, the determined head group and tail group component volumes were used in the refinement of the custom model against the NR measurements. Due to the lack of contrast present between the hydrogenous PC head group and the solvent, it was necessary to constrain the thickness of the head layer based on that determined from the lowest surface concentration XRR measurement. Since the NR measurements did not make use of a Langmuir trough sample environment, the $\phi_t$ parameter was allowed to have non-zero values. These values can be rationalised as a reduction of the compression of the tail components. To ensure that the number denisty of the head groups and pairs of tail groups was constant, the following constraint was applied to the value of $\phi_t$, 
\begin{equation}
\label{equ:phit}
\phi_t =  1 - \bigg(\frac{d_hV_t(1-\phi_h)}{V_hd_t}\bigg).
\end{equation}
Table \ref{tab:invariant} gives the details of the scattering length and SLDs used as invariant parameters in the custom model, while the tail chain lengths were the same as for the XRR. 

In both cases, the refinement of the custom model to the experimental data involved transforming the reflectometry calculated from the model and the data into $Rq^4$ such that the artifact of the Fresnel decay was removed, before using the differential evoulation method available to refnx from the scipy library,\cite{Jones2001} to find the parameters that gave the best fit to the data. The parameter space was then probed in a Bayesian fashion using the MCMC method available through emcee, this allowed for an estimate of the PDF associated with each parameter. In the MCMC sampling, 200 walkers were used over 1000 iterations, following an equilibration of 200 iterations.

\section{Results \& Discussion}
\begin{figure}
	\centering
	\includegraphics[width=0.48\textwidth]{figures/DLPC_all_data}
	\includegraphics[width=0.48\textwidth]{figures/DMPC_all_data}
	\includegraphics[width=0.48\textwidth]{figures/DPPC_all_data}
	\includegraphics[width=0.48\textwidth]{figures/DMPG_all_data}
	\caption{The XRR profiles, SLD profiles and PDFs of the head component volume for each of the four lipids, the lower surface concentration is shown in blue while the higher in green; (a) DLPC, (b) DMPC, (c) DPPC, (d) DMPG. The XRR profiles have been offset in the $y$-axis by an order of magnitude and SLD profiles offset in the $y$-axis by $5\times10^{-6}$ \AA$^{-2}$, for clarity. Figure files are available under MIT License.\cite{mccluskey_2018}}
	\label{fig:lipids}
\end{figure}
The custom model was co-refined across the two surface pressure measurements for each lipid. The resulting XRR profiles, associated SLD profiles and the PDF for the head group component volumes are shown in Figure \ref{fig:lipids}. Table \ref{tab:liptab} gives details of all varying parameters for each system, these are given with asymmetric uncertainies that correspond to a 95 \% confidence interval of the PDF, the full PDF plots can be found in the ESI. 
\begin{table*}
	\small
	\caption{\ The best-fit values, and associated 95 \% confidence intervals for the varying parameters in the XRR models. The values of $d_t$ were found from the appropriate values of $\theta_t$ using Eqn. \ref{equ:tl}. The values of $d_h$ were obtained from the appropriate use of Eqn. \ref{equ:ht}}
	\label{tab:liptab}
	\begin{tabular*}{\textwidth}{@{\extracolsep{\fill}}lllllllll}
		\hline
		Lipid & DLPC & & DMPC & & DPPC & & DMPG & \\
		Surface Pressure/mNm${-1}$ & 20 & 30 & 20 & 30 & 20 & 30 & 20 & 30 \\
		\hline
		$\theta_t$/$^\circ$ & \input{../output/dlpc/angle4.txt} & \input{../output/dlpc/angle5.txt} & \input{../output/dmpc/angle4.txt} & \input{../output/dmpc/angle5.txt} & \input{../output/dppc/angle4.txt} & \input{../output/dppc/angle5.txt} & \input{../output/dmpg/angle4.txt} & \input{../output/dmpg/angle5.txt} \\
		$d_t$/\AA & \input{../output/dlpc/tail4.txt} & \input{../output/dlpc/tail5.txt} & \input{../output/dmpc/tail4.txt} & \input{../output/dmpc/tail5.txt} & \input{../output/dppc/tail4.txt} & \input{../output/dppc/tail5.txt} & \input{../output/dmpg/tail4.txt} & \input{../output/dmpg/tail5.txt} \\
		$d_h$/\AA & \input{../output/dlpc/head4.txt} & \input{../output/dlpc/head5.txt} & \input{../output/dmpc/head4.txt} & \input{../output/dmpc/head5.txt} & \input{../output/dppc/head4.txt} & \input{../output/dppc/head5.txt} & \input{../output/dmpg/head4.txt} & \input{../output/dmpg/head5.txt} \\
		$V_t$/\AA$^3$ & \input{../output/dlpc/vt.txt} & \input{../output/dlpc/vt.txt} & \input{../output/dmpc/vt.txt} & \input{../output/dmpc/vt.txt} & \input{../output/dppc/vt.txt} & \input{../output/dppc/vt.txt} & \input{../output/dmpg/vt.txt} & \input{../output/dmpg/vt.txt} \\
		$V_h$/\AA$^3$ & \input{../output/dlpc/vh.txt} & \input{../output/dlpc/vh.txt} & \input{../output/dmpc/vh.txt} & \input{../output/dmpc/vh.txt} & \input{../output/dppc/vh.txt} & \input{../output/dppc/vh.txt} & \input{../output/dmpg/vh.txt} & \input{../output/dmpg/vh.txt} \\
		$\phi_h$/$\times10^{-2}$ & \input{../output/dlpc/solh4.txt} & \input{../output/dlpc/solh5.txt} & \input{../output/dmpc/solh4.txt} & \input{../output/dmpc/solh5.txt} & \input{../output/dppc/solh4.txt} & \input{../output/dppc/solh5.txt} & \input{../output/dmpg/solh4.txt} & \input{../output/dmpg/solh5.txt} \\
		$\sigma_{t,h,s}$/\AA & \input{../output/dlpc/rough4.txt} & \input{../output/dlpc/rough5.txt} & \input{../output/dmpc/rough4.txt} & \input{../output/dmpc/rough5.txt} & \input{../output/dppc/rough4.txt} & \input{../output/dppc/rough5.txt} & \input{../output/dmpg/rough4.txt} & \input{../output/dmpg/rough5.txt} \\
		\hline
	\end{tabular*}
\end{table*}
\subsection{Effect of increasing surface pressure on monolayer thickness}
The thickness of the head and tail layers in the model, with the tail layer being a function of the chain tilt angle are given in Table \ref{tab:liptab}. This shows that as expected, and as found in previous work,\cite{Mohwald1990,Vaknin1991} the thickness of the tail layer increases as the number of carbon atoms in the tail chain increases. The thickness of the tail layers in these condensed monolayers appears to agree well with values found for water-analogues (DMPC: $d_t=15.8$ \AA,\cite{Johnson1991} DPPC: $d_t=16.7$ \AA\cite{Helm1987}).

It is also observed that for all lipids, when the surface pressure increases there is an observed increase in the tail layer thickness, resulting from the chain tilt angle descreasing. Showing the chains to be more closely aligned with the surface normal as the surface pressure increases. The phenomena of the tail thickness increasing with increasing surface pressure has been noted before for DMPC at the air-water interface.\cite{Bayerl1990}

Alongside the increase in the tail thickness there is also a small decrease in the head layer thickness with increasing surface pressure noted for all lipid. Previous work indicates, although there has been no direct comparison, that the PG head layer may be thicker than PC at a water-air interface.\cite{Clifton2012,Johnson1991,Vaknin1991,Lawrie2000} However, this is not observed at the air-DES interface with the PG head layer thickness being similar to that for PC.

\subsection{Effect of compression on the lipid tail component volumes}
Using the Langmuir trough sample environment, it was possible to conduct XRR measurements at surface pressures of 20 mNm$^{-1}$ and 30 mNm$^{-1}$. These are anticipated to be well into the liquid condensed phase for the lipid tail groups, and therefore it would be anticipated, based on the work of Campbell and co-workers,\cite{Campbell2018} that there would be an observed reduction in the component volume of the tail groups compared to those found in the liquid expanded phase. 

It is clear when comparing Tables \ref{tab:water} and \ref{tab:liptab} that the component volumes of the tail groups are reduced in the current XRR measurements compared to those determined previously of systems in the liquid expanded phase. The reduction was found to be between 5 to 12 \% for DPPC, between 1 and 10 \% for DMPC and 8 \% for DMPG, depending on the source of the tail component volume in the liquid expanded phase. From this we feel that it is clear that the monolayers are indeed in the liquid condensed phase for all of the XRR measurements. 

\subsection{Solvent effect on lipid head component volumes}
Figures \ref{fig:lipids}(a-d) show the PDFs determined for the head group component volume for each of the four lipids. The three lipids with the PC head group are consistant with values of $\sim340$ \AA$^3$ irregardless of tail component. This agrees well with the values found for the same head group in water, shown in Table \ref{tab:water}. Interestingly the component volume for the PG head group is similar to that for the PC head group with a value of \input{../output/dmpg/vh.txt} \AA$^3$, whereas it is considered to be significantly smaller in water. This indicates that there is some affect arising from the solvation in DES causing an apparent increase in the PG component volume when compared with water. 

\begin{figure}
	\centering
	\includegraphics[width=0.30\textwidth]{figures/head_groups}
	\caption{The two lipid head groups compared in this study, where R indicates the carbon tail; (a) phosphatidylglycerol, (b) phosphocholine. Figure files are available under MIT License.\cite{mccluskey_2018}}
	\label{fig:heads}
\end{figure}

Figure \ref{fig:heads} gives the chemical structure for the two head groups investigated in this work. The major difference between the two components if the fact PG component is negatively charged whereas the PC component is zwitterionic. It has been shown previously that the conformation for the PC component is slightly folded in water,\cite{Gilliams2016} due to the interaction between the positively-charged ammonium and the negatively-charged phosphate groups. It would be fair to assume that a similar structure may occur for the PG component, with the interaction between the partially postively-charged alcoholic carbon atoms and the negatively-charged phosphate group. However, clearly such an interaction would be weaker than that observed in the PC component. We believe that the observed increase found for the PG component volume in DES when compared with water is due to the unfolding of the component. This unfolding is made possible by the charged nature of the solvent providing a greater screening effect for charges in the PG component that are present in water. This effect is not observed for PC component due to the greater strength of the folding-charges arising from the formally-charged nature of the ammonium group. 

\subsection{Response of interfacial roughness to surface pressure change}
A single value was fit for the roughness at each of the three intrefaces in the system; solvent-head, head-tail, tail-air, the value for each system is given in Table \ref{tab:liptab}. It is clear that there is an increase in the roughness observed for each of the lipid when the surface pressure is increased. This is in agreement with previous experimental work investigating the effect of surface pressure on surfactant monolayers at the air-water interface.\cite{Campbell2018,Tikhonov2000}

\subsection{Refinement of neutron reflectometry}
The custom model was refined individually for each of the 4 NR measurements, two contrasts for each of the two lipids studied. The resulting reflectometry profiles and associated SLD profiles are given in Figure \ref{fig:neutron}. Table \ref{tab:neutron} gives details of all of the varying parameters for each measurement, again these are given with a asymmetric uncertainties corresponding to a 95 \% confidence interval and the full PDF plots can be found in the ESI.

\begin{figure}
	\centering
	\includegraphics[width=0.36\textwidth]{figures/nDMPC_all_data}
	\includegraphics[width=0.36\textwidth]{figures/nDPPC_all_data}
	\caption{The NR and SLD profiles for each of the four lipids, the h-DES contrast is shown in blue while the hd-DES in green; (a) DMPC, (b) DPPC. The NR and SLD profiles have been offset in the $y$-axis for clarity. Figure files are available under MIT License.\cite{mccluskey_2018}}
	\label{fig:neutron}
\end{figure}

The ability to fit the NR data, as shown in Figure \ref{fig:neutron} indicates that the values found for the head group component volume is accurate for the given systems. Additionally, it is also clear from the magnitude of the $\phi_t$ parameter that despite the best effort it was not possible to form condensed phase monolayer of the phospholipid without the force available to a Langmuir trough. This is evident from the relatively high volume fractions of air in the tail layer and the high chain tilt angles adopted by the tails. 

\begin{table}
	\small
	\caption{\ The best-fit values, and associated 95 \% confidence intervals for the varying parameters in the NR models. The values of $d_t$ were found from the appropriate values of $\theta_t$ using Eqn. \ref{equ:tl}}
	\label{tab:neutron}
	\begin{tabular*}{0.48\textwidth}{@{\extracolsep{\fill}}lllll}
		\hline
		Lipid & d$_{54}$-DMPC & & d$_{62}$-DPPC & \\
		Solvent & h-DES & hd-DES & h-DES & hd-DES \\
		\hline
		$\theta_t$/$^\circ$ & \input{../output/dmpc/angle3_neutron_n1.txt} & \input{../output/dmpc/angle3_neutron_n2.txt} & \input{../output/dppc/angle3_neutron_n1.txt} & \input{../output/dppc/angle3_neutron_n2.txt} \\
		$d_t$/\AA & \input{../output/dmpc/tail3_neutron_n1.txt} & \input{../output/dmpc/tail3_neutron_n2.txt} & \input{../output/dppc/tail3_neutron_n1.txt} & \input{../output/dppc/tail3_neutron_n2.txt} \\ 
		$\phi_t$/$\times10^{-2}$ & \input{../output/dmpc/solt3_neutron_n1.txt} & \input{../output/dmpc/solt3_neutron_n2.txt} & \input{../output/dppc/solt3_neutron_n1.txt} & \input{../output/dppc/solt3_neutron_n2.txt} \\
		$\phi_h$/$\times10^{-2}$ & \input{../output/dmpc/solh3_neutron_n1.txt} & \input{../output/dmpc/solh3_neutron_n2.txt} & \input{../output/dppc/solh3_neutron_n1.txt} & \input{../output/dppc/solh3_neutron_n2.txt} \\
		$\sigma_{t,h,s}$/\AA & \input{../output/dmpc/rought3_neutron_n1.txt} & \input{../output/dmpc/rought3_neutron_n2.txt} & \input{../output/dppc/rought3_neutron_n1.txt} & \input{../output/dppc/rought3_neutron_n2.txt} \\
		\hline
	\end{tabular*}
\end{table}

\section{Conclusions}
Stable phosphocholine and phosphatidylglycerol lipid monolayers have been observed at the air-DES interface, and chemically-relevant modelling and Bayesian analysis was used to rationalise XRR measurements, allowing for the quantification of the effec that the non-aqueous DES had on their structure. The structure of the PC component containing lipids was found to be very similar at the air-DES interface to that at the air-water interface. However, the PG component containing lipid was found to have a significantly larger head component volume than observed for the same system in water. We propose that this is due to the unfolding of the PG head component arising from the electrostatic screening of the component charges due to the presence of the charged DES solvent. This unfolding may not occur for the PC component to the strong interaction arising from the formal charge of the ammonium group. 

The ability to determine the head group volume was facilitated by access to easy to use, and open-source software that allowed for the straightforward use a custom, chemically-consistant model to be used within the analysis of the XRR and NR measurements. Futhermore, this work presents the first, to our knowledge, use of chemically-consistant parameterisation to co-refine XRR measurements at different surface concentrations. 

Until the emergence of ionic liquids and DES, only a limited number of molecular solvents exhibited the ability to promote self-assembly and, to the best of our knowledge, only water among those had demonstrated the formation of functional phospholipid monolayers at the air-liquid interface. Therefore, choline chloride:glycerol DES constitutes a novel environment where phospholipid membranes may be investigated. These possibilities include fundamental investigations of phospholipid monolayers in extreme environments (total or partial absence of water, cryogenic temperatures), protein membrane interactions and development of new technologies for drug delivery.

\section*{Conflicts of interest}
There are no conflicts to declare.

\section*{Acknowledgements}
The authors would like to thank the European Spallation Source and the University of Bath Alumni Fund for supporting A.S.-F. A.R.M. is grateful to the university of Bath and Diamond Light Source for co-funding a studentship (Studentship Number STU0149). We also thank Diamond Light Source (SI10546-1) and Institut Laue-Langevin for the awarded beamtime (DOI: \href{http://doi.org/10.5291/ILL-DATA.9-13-612}{10.5291/ILL-DATA.9-13-612}).

%%%END OF MAIN TEXT%%%

%The \balance command can be used to balance the columns on the final page if desired. It should be placed anywhere within the first column of the last page.

\balance

%If notes are included in your references you can change the title from 'References' to 'Notes and references' using the following command:
%\renewcommand\refname{Notes and references}

%%%REFERENCES%%%
\bibliography{rsc} %You need to replace "rsc" on this line with the name of your .bib file
\bibliographystyle{rsc} %the RSC's .bst file

\end{document}
