\documentclass[amsmath,amssymb,twocolumn,superscriptaddress]{revtex4-1}

\usepackage{graphicx}% Include figure files
\usepackage{dcolumn}% Align table columns on decimal point
\usepackage{bm}% bold math
\usepackage[detect-all]{siunitx}
\usepackage{hyperref}% add hypertext capabilities
\usepackage{xr}
\externaldocument{si}

\begin{document}

\title{Referees reports for ``Bayesian determination of the effect of a deep eutectic solvent on the
structure of lipid monolayers''}% Force line breaks with \\%

\author{A.~R.~McCluskey}
\thanks{A.R.M. and A.S.-F. contributed equally to this work}
\email{a.r.mccluskey@bath.ac.uk/andrew.mccluskey@diamond.ac.uk}
\affiliation{Department of Chemistry, University of Bath, Claverton Down,
Bath, BA2 7AY, UK}
\affiliation{Diamond Light Source, Harwell Campus, Didcot, OX11 0DE, UK}

\author{A.~Sanchez-Fernandez}
\thanks{A.R.M. and A.S.-F. contributed equally to this work}
\altaffiliation[Present address: ]{Department of Food Technology, Lund
University, SE-211 00 Lund, Sweden.}
\affiliation{Department of Chemistry, University of Bath, Claverton Down,
Bath, BA2 7AY, UK}
\affiliation{European Spallation Source, SE-211 00 Lund, Sweden}

\author{K.~J.~Edler}
\affiliation{Department of Chemistry, University of Bath, Claverton Down,
Bath, BA2 7AY, UK}

\author{S.~C.~Parker}
\affiliation{Department of Chemistry, University of Bath, Claverton Down,
Bath, BA2 7AY, UK}

\author{A.~J.~Jackson}
\affiliation{European Spallation Source, SE-211 00 Lund, Sweden}
\affiliation{Department of Physical Chemistry, Lund University, SE-211 00
Lund, Sweden}

\author{R.~A.~Campbell}
\affiliation{Division of Pharmacy and Optometry, University of Manchester,
Manchester, UK}
\affiliation{Institut Laue-Langevin, 71 avenue des Martyrs, 38000, Grenoble,
France}

\author{T.~Arnold}
\email{tom.arnold@esss.se}
\affiliation{Department of Chemistry, University of Bath, Claverton Down,
Bath, BA2 7AY, UK}
\affiliation{Diamond Light Source, Harwell Campus, Didcot, OX11 0DE, UK}
\affiliation{European Spallation Source, SE-211 00 Lund, Sweden}
\affiliation{ISIS Neutron and Muon Source, Science and Technology Facilities
Council, Rutherford Appleton Laboratory, Harwell Oxford, Didcot, OX11 0QX,
UK}

\date{\today}% It is always \today, today,
             %  but any date may be explicitly specified

\maketitle

Note that the Journal names have been redected.

\section{Submission to Journal A: Rejection following peer review}

\subsection{Referee 1}
I am afraid I failed to see the special value in this manuscript
required for Journal A.

\subsection{Referee 2}
I don't think this is appropriate for Journal A.
In particular, the new Bayesian method requires more detail that should be included in a full paper.
A full paper in a specialty journal like Journal B would be better.

Full article - I also suggest that Journal C is not the best journal for this paper.
Journal B would be better.

\subsection{Referee 3}
This manuscript describes an examination of the structure of the insoluble monolayers formed by four phospholipids at the air/deep eutectic solvent interface by X-ray and neutron reflectometry.
The work reported examines a novel class of systems and finds some unexpected results, as well as utilizing some novel methodology.
Deep eutectic solvents are an important emerging class of liquids related to ionic liquids, with broad interest across the physical and material sciences.

The work is carefully done and conclusion reliable as far as I can ascertain.
However the authors should consider some minor revisions as follows before proceeding to publication.

\begin{enumerate}
\item It is incorrect to claim that these are the first studies of insoluble monolayers at an air/non-aqueous liquid interface. Lipid monolayers on formamide have been reported previously e.g. 10.1051/jp2:1995131 and 10.1021/jp9805412.

\item I could not find a clear description of the isotopic compositions of the components used for NR. ``hDES'' is unambiguous, but ``hdDES'' could mean many things. It may be detailed in the ESI but should be in the main paper.

\item Consider including the main ESI document (``si'' on GitHub) as a separate link. The extensive supplementary information provided including data files, is excellent, but specifically called-out references to ESI should be more readily accessible without searching GitHub or the Bath University repository dataset.
\end{enumerate}

The major outcomes of this work are the Bayesian methodology development and the unexpected head group volume of PG compared to water. The significance of this is not clear, and perhaps requires more exposition to justify publication in Journal A.

\subsection{Referee 4}
This paper describes a well-executed structural study of lipid monolayers assembled at an air/non-aqueous liquid interface based on x-ray and neutron reflectivity measurements. The results are solid, and there are many positive qualities to this study which will benefit researchers in the relevant field. However, I do not think that this paper merits a publication in Journal A mainly because the generality and robustness of the main scientific result (i.e., increased volume of the DMPG head group in DES relative to that in water) is uncertain, as explained further below. In my opinion, the study falls short of making a compelling case that it meets the ``broad interest'' criterion of the Journal A.

There are three potentially novel aspects to the study described in this paper:

\begin{enumerate}
\item Self-assembly of phospholipid monolayers at the free surface of an non-aqueous liquid, DES in the present work;

\item Application of a Bayesian approach to the analysis of x-ray and neutron reflectivity results;

\item Evidence for an apparent increase in the volume of DMPG head group at the DES interface, presumably due to ``unfolding of the PG head group,'' as compared to that at the water interface.
\end{enumerate}

For 1., the conclusion about the stable formation of the lipid monolayers at the DES surface is very convincing, and the authors’ claim about ``the first observation of phospholipid monolayers at an air-DES interface (or for that matter, any non-aqueous media)'' may be technically correct. But at the same time, the observation is not surprising, given that (i) phospholipids are prime examples of amphiphilic surfactant molecules, and other amphiphilic surfactant molecules have been shown to form stable monolayers at air-DES interfaces (e.g., 10.1021/acs.langmuir.5b02596, 10.1063/1.4952444); and (ii) as noted by the authors, phospholipids in bulk DES have been shown to exhibit qualitatively similar behavior as those in bulk water, such as formation of bilayers and vesicles (10.1021/acs.langmuir.7b01561, 10.1039/C5SM02660A]). Moreover, formation of stable molecular monolayers on non-aqueous liquid surface has previously been observed, as demonstrated for example by a series of previous x-ray reflectivity-based studies of monolayers on mercury surface, by M. Deutsch, B. Ocko, P. Pershan, O. Magnussen, et al. [e.g., Magnussen et al., Nature 384, 250 (1996); Kraack et al., Science 298, 1404 (2002)].

For 2., the application of a Bayesian approach to the reflectivity analysis is well executed, and the authors do a good job of highlighting its utility, especially in illustrating the correlations between fitting parameters. The practitioners of x-ray and neutron reflectivity analysis will greatly benefit from these illustrative results. However, the novelty and potential impact of (ii) is technical rather than scientific.

In my opinion, aspect 3. of this study’s results is the most scientifically interesting. However, the evidence for the increased head-group volume in DES as compared to that in water is obtained for one particular molecule (DMPG) at the surface of one particular DES. Moreover, the interpretation for the physical origin of the increased head volume, i.e., unfolding of the PG group in DES as compared to the conformation in water, remains speculative and unsubstantiated (e.g., either experimentally or through simulation). As such, the conclusion about the increased PG head volume could be system-specific, and its generality and robustness is far from being well established. Demonstration of generality and robustness would require that similar evidence be gained for a series of PG-containing phospholipids and other types of lipid head groups with relatively weak intra-head interactions (like in PG as opposed to the zwitterionic PC’s).

On the basis of the above, I recommend that this paper be submitted elsewhere to a more specialized journal if its content is to remain as it is. Strengthening the generality of the study’s main scientific result and substantiation of the proposed ``unfolding'' mechanism should be a prerequisite before the manuscript is considered further for publication at the Journal A.

Additional specific comments:

\begin{itemize}
\item Use of ``chemically consistent'' constraints used makes sense and is very nice. Although fixing head-layer thickness $d_{h}$ for all surface pressures measured doesn’t have a good physical justification, it may not be such a serious offense given that this parameter usually has large error bars and doesn’t seem to show any trend with surface pressure that goes outside of the error bars (e.g., 10.1021/jz9002873).

\item Given that the LE phase consists of tail chains that are not stretched out as all-trans, it does not make sense to talk about the tail tilt angle $\theta_{t}$ for DLPC, DMPC, and DMPG. It is suggested that the analysis use the tail-layer thickness $d_{t}$, instead of $\theta_{t}$, as the freely varied parameter, with the constraint that $d_{t}$ be less than the upper bound of $t_{t}$, the maximum all-trans length of the tail. The tilt angle $\theta_{t}$ can be extracted and tabulated for DPPC, which is in the LC phase.

\item The NR roughness of 2.5 Angstrom for d54-DMPC at 25 mN/m seems too small, being comparable to the capillary-wave roughness of water at room T. It is suggested that the calculated capillary roughness (which depends on surface tension, temperature, molecular size, and instrumental resolution) be used as the lower-bound constraint for the surface roughness parameter used in the fitting.
\end{itemize}

\subsection{Changes following rejection}
\begin{itemize}
\item Following the comment from Referee 3, all claims that this is the first example of a lipid monolayer on a non-aqueous solvent have been removed. Instead claiming the first example of a lipid monolayer on an ionic solvent. Furthmore, the references mentioned for lipid monolayers on formamide have been included and mentioned in the text.
\item In the analysis process, the use of $\theta_t$ has been replaced simply with the fitting of a tail thickness, $d_t$. This removes the (inaccurate) assumption that the lipid tails are stretched out in an all-trans conformation (Referee 4).
\item The chemically consistent model that is used is detailed fulling in the main text of the paper now (Referee 2).
\item The justification for the constraint of the head thickness is further detailed, specifically that there is little variation noted in previous work (Referee 4).
\item Inclusions of a more detailed discussion of the inter parameter correlations, that are available from the use of Markov chain Monte Carlo sampling (Referee 2).
\item The lower limit for the NR roughness was increased to 3.3, as suggested by Referee 4.
\item A more detailed description of the neutron contrasts is given in the main text (Referee 3).
\item A link has been included on the GitHub page to access the \texttt{si.pdf} document more easily (Referee 3).
\end{itemize}


\end{document}
